\documentclass{Cricketer's Health Supervision}

\title{Test}

\programmeName{AI \& DS}
\courseName{Internet of Things}
\assignment{Research Paper Summary}
\date{22 January 2023}

\author{Varshith P Singh}
\iuklId{21011101137}
\semester{4th Semester}

\facultyName{Santhi Natarajan}
\department{CSE}



\begin{document}

\maketitle
   \begin{center}
        \section*{Cricketer's Health Supervision using Advanced IoT System}
    \end{center}
\setlength{\columnsep}{1.0cm}
    \large
    \section*{Summary}
    Cricket is the most popular sport in India and also in many countries all over the world. In comparison with any sport, cricket contains a high fan following all over the world .Cricketers face many health-related issues during the match as well as in practice sessions. A lot of effort has to put by players during training and matches, especially by the bowlers who run all day in the sun, which leads to shortness in breath, hypoglycemia, and many more. Many sponsors and Cricket committees spend lot of money on cricketers. Health issue of any player gives a huge blow to their spending on the individual.\\
    
    This paper discusses an advanced IoT support methodology
    which monitors the health of a player by using embed
    sensing devices, telecommunication technologies and cloud computing. Any case of injuries, breathing issues, or any kind of health issue is immediately analyzed and resolved. Sensors are attached to the body of the players during the match or any practice sessions. Parameters like heart rate, sweat rate, and respiration rate are collected by the sensors attached to the body of the players. The collected data is transmitted to the cloud, then forwarded to the physician through
    mobile device which provides information on whether a player needs medical attention.\\
    
     For example, in case a batter has been struck on the helmet by the ball, the physio rushes to the ground to check if the batter has had a concussion. But, this IoT system provides a much better approach where the heart rate, sweat rate, and respiration rate are recorded. Therefore, in case the physio feels that the batter needs to be off the field, a call can be taken on that. 
     This system basically tries to minimize the physical health risks that are involved in the game.
    
    \section*{Key technologies discussed by the author}
     The author says that in order to uniquely identify a player, an \textbf{RFID} is used to provide information about that player. It consists the player's ID \& medical history. He explains in brief the 3 layer architecture that is employed in the Cricket IoT System. The following are the functions performed by each layer:
    
    \begin{enumerate}
        \item \textbf{Perceptron Layer} - Consists of 2 networks of sensors: \textbf{WBANSs (wireless area networks)} \& \textbf{RPL (Routing protocol for low power)}. These are special kinds of networks which have the ability to measure physiological parameters of a player. These sensors are less in weight and can be tied to the player's body.
        \item \textbf{Network Layer} - The signals recieved from the perceptron layer is sent to the cloud using the \textbf{ZigBee} technology.
        \item \textbf{Application Layer} - A cloud layer analyzes the data traffic which comes from devices for giving accurate feedback to the physician. The support of predictive analytics in data mining is used for analyzing, predicting the health conditions of the cricketers.
    \end{enumerate}
    
    \section*{My views about the paper}
    In my opinion, this paper shows a very good approach to detect any kind of physical difficulties that the players experience during the game. There is a certain instance, where a batter was struck near the back of the neck \& he fell unconscious the very next moment on the field. In that case, the medics took him out of the field and admitted him in a hospital, but sadly, he was declared dead. So, in cases like these, it would be better it they could know the batter's heart rate, sweat rate, and respiration rate, which would help the medics take the players off the field. Some players, though, they suffer with difficulties on the field, when the team physio requests them to come off the field, the players refuse, though that's great determination to play for the , it involves a lot of risk. Whereas if the physio could track their physiological parameters all the time, it would help them take the right decision on whether it is good for the player to stay on the field or to take him/her off the field for medical attention.\\

     I appreciate the fact that a players physical health is being taken into consideration, but personally I would love to see some steps being taken towards monitoring the mental health of players.\\
    
    Poor performances are harder to hide when it comes to a team sport which is capable of creating a huge amount of self-induced pressure, the pressure of wanting to do well, and a high fear of failure. It creates insecurity, with the constant worries of whether the player's contribution is being valued and the questioning of their own ability. All of these things, or a combination of them, are capable of leading to depression, which should be taken care of at the earlier stages. Developing systems to track a player's energy levels, concentration, their sleep cycle \& other symptoms related to depression would be the next step from my perspective, as some top players of the game go off taking indefinite breaks due to the workload \& pressure involved.
    
    \section*{Agreement, Pitfalls and Fallacies}
    I appreciate the idea that has been put forth and it would be a step in the right direction if it is successfully implemented.But, there are some obstacles that cannot be avoided, which are as follows:
    
    \begin{itemize}
        \item The technology used or components used may affect the infrastructure of the cricket stadium. 
        \item Cricketers may not compromise in sharing the health details as they may have a perception that their privacy can be breached.
        \item Intercommunication between the devices must be of high level consistency and functionality in order to give the best results, thereby making this IoT system a success.
        \item Performance levels of a communication system must be improved from time to time.
    \end{itemize}
\end{document}
